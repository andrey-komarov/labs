\section{Задание}
\subsection{Вариант 8}

Выберите подмножество теха ии напишите его конвертор в HTML.
При необходимости используйте MathML.

\section{Ход работы}
\subsection{Выбор нужного подмножества}

В качестве подмножества теха было выбрано нечто, с помощью чего можно было 
написать этот отчёт, а также немного формул.

\subsection{Используемые инструменты}

Для вывода математических формул на html страницу используется MathML: теги
mrow, mfrac, mo, mi, mn и, вроде бы, какие то ещё.

\subsection{Проблемы}

Во время выполнения этой 

\section{Примеры}

\subsection{Число, переменная}

$2$, $1234$, $x$, $y$, $adsf$

\subsection{Операции}

$1 + 1$, $2 - 2 - 2$, $5 * 5$, $1/   0$

\subsection{Отношения}

$1 < 2$, $2 + 2 = 5$, $2 > 1$, $2 \ne 1$, $100 \ge 10$, $1 \le 10$

\subsection{Верхний и нижний индексы}

$2^2$, $x_1$, $x_1^2$, $x^2_1$

К сожалению, последние два выглядят весьма уныло, хотя должны были бы выглядеть
одинаково и красиво. но я только сейчас осознал, что для того, чтобы это реализовать,
нужно использовать тег msubsup, который, как следует из названия, принимает
аргументы в порядке: нижний индекс, затем верхний индекс. А для того, чтобы это
обеспечить, необходимо уметь менять порядок обхода дерева разбора, чего с 
Listenerом, с которым написан весь код, я делать не умею и не уверен, можно ли
вообще, и нужно переписать на Visitor, а мне лень.

\subsection{frac}

$\frac{1}{2}$, $\frac{x}{2}$, $\frac{123 + x^2}{y_2}$

\subsection{Всё вместе}

${[2^{(\frac{\lfloor 1 \rfloor}{\lceil x \rceil})} = 0]}_{\{z\}} - 
\frac{1}{2 + \frac{3}{4 + \frac{5}{a}}}$

\section{Примечание}

Этот файл можно собрать TeXом. И на этом файле работает программа.
