\section{Разработка грамматики}

\subsection{Задание}
Описание переменных в Паскале.

\subsection{Построение грамматики}
\begin{tabular}{ r c l }
    $S$ & $\to$ & $\mathrm{var}\, D$ \\
    $D$ & $\to$ & $E\, D\, |\, \varepsilon$ \\
    $E$ & $\to$ & $\mathrm{n}\, I\, \mathrm{:\, n\, ;}$ \\
    $I$ & $\to$ & $\mathrm{,\, n}\, I\, |\, \varepsilon$ \\
\end{tabular}

\begin{center}
    \begin{tabular}{ | l | l | }
        \hline
        \textbf{Нетерминал} & \textbf{Описание} \\
        \hline
        $S$ & Начальный нетерминал. \\
        \hline
        $D$ & Список определений. \\
        \hline
        $E$ & Одно определение. \\
        \hline
        $I$ & Список переменных. \\
        \hline
    \end{tabular}
\end{center}

В грамматике нет левой рекурсии или правого ветвления. 

