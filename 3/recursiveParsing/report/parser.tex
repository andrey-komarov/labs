\section{Построение синтаксического анализатора}

Построим множества \textbf{FIRST} и \textbf{FOLLOW} для нетерминалов.

% S -> "var" D
% D -> ED | eps
% E -> n I : n ;
% I -> , n I | eps

\begin{tabular}{| l | l | l |}
    \hline
    \textbf{Нетерминал} & \textbf{FIRST} & \textbf{FOLLOW} \\
    \hline
    $S$ & n(<<var>>) & n, \$ \\
    \hline
    $D$ & n, $\varepsilon$ & n, \$ \\
    \hline
    $E$ & n & n, \$ \\
    \hline
    $I$ & n, $\varepsilon$ & : \\
    \hline
\end{tabular}
\\
Структуры для хранения дерева полностью повторяют грамматику:

\begin{verbatim}
data SNode = SNode DNode
data DNode = DNode ENode DNode | DEps
data ENode = ENode String INode String
data INode = INode String INode | IEps
\end{verbatim}

